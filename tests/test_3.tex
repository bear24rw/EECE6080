\documentclass[9pt,fleqn]{article}
\usepackage[margin=0.5in]{geometry}
\usepackage{multicol}
\usepackage{lipsum}% dummy text
\usepackage[fleqn]{amsmath}
\usepackage[compact]{titlesec}
\usepackage{graphicx}

\setlength{\mathindent}{0pt}
\setlength{\columnseprule}{0pt}
\setlength{\parskip}{0pt}
\setlength{\parindent}{0pt}
\setlength\abovedisplayskip{0pt}
\setlength\belowdisplayskip{0pt}
\setlength{\jot}{0pt}
\linespread{0.9}

\allowdisplaybreaks

\begin{document}
\begin{multicols}{3}

    \textbf{Demorgans Law}
    \begin{flalign*}
        \neg[p\wedge q]\equiv\neg p\vee\neg q,\\
        \neg[p\vee q]\equiv\neg p\wedge\neg q.
    \end{flalign*}
    \textbf{Rise/Fall time}
    \begin{flalign*}
        t_{r} = R_{p}C ,
        t_{f} = R_{n}C \\
        R_{n} \propto {L_{n} \over W_{n}\mu_{0_{n}}},
        R_{p} \propto {L_{p} \over W_{p}\mu_{0_{p}}}
    \end{flalign*}
    \textbf{Elmore's delay}
    \begin{flalign*}
        &\tau_{Di} = \sum_{k=1}^{N}{ R_{k} \sum_{j=k}^{N} {C_{j} }} \\
        %&\tau_{Di} = \sum_{k=1}^{N} C_{k}*R_{ik} \\
        &Estimated Delay = \tau_{p} = 0.69*\tau_{Di} \\
        &Total Expected Energy = C_{L}*f*{{V_{dd}}^2}
    \end{flalign*}
    \textbf{Power Probability}
    \begin{flalign*}
        P_{n\_nand}(0) &= P_{A}(1)P_{B}(1) \\
        P_{p\_nand}(1) &= 1-P_{A}(1)P_{B}(1) \\
        P_{n\_nor}(0)  &= 1-P_{A}(0)P_{B}(0) \\
        P_{p\_nor}(1)  &= P_{A}(0)P_{B}(0)
    \end{flalign*}
    \textbf{Buffer Chain} \\
    First buffer is of size 1
    \begin{flalign*}
        N &= Total\ numbers\ of\ buffers\ \\
          &Size\ of\ buffer\ i = {C_{L} \over C_{1}}^{i \over N} \\
        f &= \sqrt[N]{{C_{L} \over C_{1}}}
    \end{flalign*}
    \textbf{Min Path Delay}
    \begin{flalign*}
        t_{p} = N*t_{po}*(1 + f)
    \end{flalign*}

    \textbf{Mux-Based Latch}
    \includegraphics[width=\linewidth]{not.png}
    \begin{flalign*}
        Q = Clk Q + ~Clk D
    \end{flalign*}
    \textbf{Inversion Property}
    Inverting all the inputs and ouputs
    \begin{flalign*}
        ~S(A,B,C) = S(~A,~B,~C)
    \end{flalign*}
    Minimize critical path by reducing inverting stages
    \textbf{Binary Adder}
    \begin{flalign*}
        S &= (A ~B ~C_{i})+(~A B ~C_{i})+(~A ~B C_{i})+(ABC) \\
    C_{o} &= AB + BC_{i} + AC_{i} \\
        P &= A ^ B \\
        G &= AB \\
        C_{o}(G,P) &= G+PC_{i} \\
            S(G,P) &= P ^ C_{i}
    \end{flalign*}
    \textbf{Ripple-Carry Adder}
    Worst case delay linear with number of bits
    \begin{flalign*}
        t_{d} = O(N)
        t_{adder} = (N-1)t_{carry}+t_{sum}
    \end{flalign*}
    \textbf{Mirror Adder}
    Maximum of 2 transistions in carry gen circuit\\
    Node Co has a lot of capacitance\\
    Only transistors in carry chain have to be optimized for size. \\
    Transistors in sum stage can be minimal size. \\
    \includegraphics[width=\linewidth]{mirror_adder.png}
    \textbf{Carry-Bypass Adder}
    In worst case carry generatored at the first bitpositon skips around the
    bypass stages and is consumed in the last bit position without generating
    an output carry.
    \includegraphics[width=\linewidth]{carry_bypass.png}
    \begin{flalign*}
        t_{adder} = t_{setup}+M_{carry}+(N/M-1)t_{bypass}+(M-1)t_{carry}+t_{sum}
    \end{flalign*}
    \textbf{Carry-Bypass Adder}
    \begin{flalign*}
        t_{adder} = t_{setup}+M*t_{carry}+\sqrt{2N}*t_{max}+t_{sum}
    \end{flalign*}
    \textbf{Linear Carry Select}
    \begin{flalign*}
        t_{adder} = t_{setup}+M*t_{carry}+(N/M)*t_{mux}+t_{sum}
    \end{flalign*}
    \textbf{Log Look Ahead Adder}
    \begin{flalign*}
        t_{p} ~ log_{2}(N)
    \end{flalign*}
    \textbf{Fast Complex Gate Design Techniques -} - Progressive sizing.
    -Distributed RC line. Can Reduce Delay by more than 20 percent. decreasing
    gains as technology stinks.\\
    \includegraphics[width=\linewidth]{ordering.png}
    -Alternative logic structures -Isolating fan-in from fan-out using buffer insertion\\
    \textbf{Ratioed Logic -} Reduce the number of devices over complementary CMOS
    \begin{flalign*}
        V_{OH} &= V_{DD}\\
        V_{OL} &= R_{PN}/(P_{PN}+R_{L})
    \end{flalign*}
    \textbf{Pass Transistor Logic} \\
    \includegraphics[width=\linewidth]{pass.png}
    \textbf{Clock Feed Through} \\
    \includegraphics[width=\linewidth]{clock_feed.png}
    \textbf{Solution to charge redistribution} \\
    \includegraphics[width=\linewidth]{charge.png}

    \includegraphics[width=\linewidth]{not.png}
    \includegraphics[width=\linewidth]{and.png}
    \includegraphics[width=\linewidth]{or.png}
    \includegraphics[width=\linewidth]{nor.png}
    \includegraphics[width=\linewidth]{xor.png}
    \includegraphics[width=\linewidth]{dynamic_gates.png}

    \textbf{Inertial Delay -} Time it takes for signal to change value \\
    \textbf{Transport Delay -} Time it takes for signal to travel down wire \\
    \textbf{Delta Cycle -} Is used to order events in VHDL simulation and
    refers to a zero-physical time. The kernel takes an additional cycle (of
    delta advancement) to update the evaluated 'future value' into the 'current
    value' registers, during which the clock doesn't advance. \\
    \textbf{Channel Length Modulation -} Shorting of the length of the inverted
    channel region with increase in drain bias for large drain biases \\
    \textbf{Velocity Saturation -} Carrier velocity reaches maximum value in
    presence of electric field. \\
    \textbf{Latch up -} A short circuit in which a low impedance path is
    created resulting in a parasitic subcircuit that disrupts proper function.\\
    \textbf{Velocity Saturation -} when a strong enough electric field is
    applied, the carrier velocity in the semiconductor reaches a maximum value.
    As the applied electric field increases from that point, the carrier
    velocity no longer increases because the carriers lose energy through
    increased levels of interaction with the lattice, by emitting phonons and
    even photons as soon as the carrier energy is large enough to do so.\\
    \textbf{Hot Carrier -} Either holes or electrons  that have gained very
    high kinetic energy after being accelerated by a strong electric field in
    areas of high field intensities within a semiconductor .  Because of high
    kinetic energy they can get injected and trapped in areas of the device
    where they shouldn't be.\\
    \textbf{Scan Chain -} Is a technique used for testing digital circuits. The
    flip flops are all connected together as a shift register into a single (or
    multiple) scan chains. It increases observability and controllability in
    the design and also helps to reduce the size of the test-set vector. \\
    \textbf{Process gain factor -} $$ K = {\mu \epsilon_{iO_{2}} \over t_{ox}} $$
    \textbf{Intrinsic Capacitance -} refers to the internal capacitance of a
    static gate generated dude to diffusion layers, metal contacts, poly-wires.
    The parasitic capacitance is proportional to the dimensions of the gate
    (i.e. W and L of transistor).  As we increase the size of the gate, the
    intrinsic capacitance increases, in turn increasing the delay of the gate.
    We aim to reduce intrinsic capacitances, but it can never be made '0'.\\
    \textbf{TSPCR -} True Single Phase Clock Register performs the flip-flop
    operation with little power and at high speeds. Stores output using
    capacitance. Will typically not work at static or low clock speeds: given
    enough time, leakage paths may discharge the parasitic capacitance enough
    to cause the flip-flop to enter invalid states.\\
    \textbf{Master Slave FF -} Is edge trigged. Two gated D latches in series
    and and inverting the enable input to one of them.\\
    \textbf{Inverter Threshold -} When the inverter switches value. NMOS pulls
    the output low, PMOS pulls output high. To increase the threshold,
    increase the strength of NMOS and/or reduce strength of PMOS. Strengh of
    MOSFET is W/L (width/length) ratio. Bigger radio = stronger MOSFET.\\
    \textbf{Method to overcome leakage -} Install keepers at the output of
    circuit. This uses pos feedback to reduce leakage. Keeper = inverter in
    series with output with output of inverter connected to input of inverter.\\
    \textbf{Clock Skew -} Spatial variation in temporally equivalent clock
    edges; deterministic + random, t\_sk. \\
    \textbf{Clock Jitter -} Temporal variations in consecutive edges of the
    clock signal; modulation + random noise. Cycle to Cycle (short-term) t\_js.\\
    \textbf{Dealing with Capacitive Cross Talk -} Avoid floating node. Protect
    sensitive nodes. Make rise fall times large. Do not run wires together for
    long dist. Use shielding for wires and layers. \\
    \textbf{C2MOS Pipeline -} Racefree as long as even number of inverters. \\
    \textbf{TSPCL -} True Single Phase Clock Logic, simplification of NORA so
    single clock phase is sufficient.\\



\end{multicols}
\end{document}
